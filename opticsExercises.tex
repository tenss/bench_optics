%%%%%%%%%%%%%%%%%%%%%%%%%%%%%%%%%%%%%%%%%%%%%%%%%%%%%%%%%%%%%%%%%%%%%%%
%%%%  Load the document class and packages                         %%%%
%%%%%%%%%%%%%%%%%%%%%%%%%%%%%%%%%%%%%%%%%%%%%%%%%%%%%%%%%%%%%%%%%%%%%%%
\documentclass[a4paper]{report}
\usepackage{epsfig}            % to insert PostScript figures
\graphicspath{ 
  {figures/} 
}

%Change figure names
\renewcommand{\figurename}{Fig}

\usepackage[bf,footnotesize]{caption} % make captions small and label bold


\addtocounter{chapter}{1} %Because starting at zero is silly
\makeatletter
\renewcommand{\thesection}{\@arabic\c@section}
\renewcommand{\thefigure}{\@arabic\c@figure}
\makeatother

\usepackage[a4paper,margin=2.7cm,tmargin=2.5cm,bmargin=2.5cm]{geometry} 
\usepackage{textcomp}          % To make nice degree symbols and others\usepackage[bf,footnotesize]{caption} % make captions small and label bold
\usepackage{wrapfig}
%to produce the clickable references along the left in Acroread. This
%package must be included last. 
\usepackage[ps2pdf,bookmarks=TRUE]{hyperref} 



%%%%%%%%%%%%%%%%%%%%%%%%%%%%%%%%%%%%%%%%%%%%%%%%%%%%%%%%%%%%%%%%%%%%%%%
%%%%  Hypertext references for Acrobat                             %%%%
%%%%%%%%%%%%%%%%%%%%%%%%%%%%%%%%%%%%%%%%%%%%%%%%%%%%%%%%%%%%%%%%%%%%%%%
\hypersetup{
pdfauthor = {TENSS},
pdftitle = {Optics Exercises},
pdfkeywords = {optics, lenses, refraction, reflection, dispersion,
  telescope, microscope},
pdfcreator = {LaTeX with hyperref},
pdfproducer = {dvips + ps2pdf}
           }


\begin{document}




%set the number of sectioning levels 
\setcounter{secnumdepth}{2}

\begin{center}
\textbf{\Large{Optics Bench Exercises}}
\end{center}

%\section{Introduction}
%These exercises are designed to teach the basic principles of optics and image formation
%\footnote{This document was originally written by Francesca Anselmi for a CSHL graduate course.
%It was then modified for use at TENSS (www.tenss.ro) by Priyanka Gupta and Adriana Dabacan. 
%The present version was created in 2016 by Rob Campbell for an imaging course at the Biozentrum, Basel.}.
%The goal of the exercises is to nurture an understanding of image formation, conjugate planes, and how multiple lenses interact in simple optical set ups. 
%It is helpful to have the free \textbf{ray optics simulator} from the Google Play store running on a computer when going through the exercises. 

%\section{Parts list}
%The assumed parts list is described here, but the same exercises can be carried out with similar parts. 
%Most of the following components are purchased from ThorLabs and imperial part numbers are listed.
%Metric parts are also available. 
%Each set up comprises the following components:
%
%
%\subsubsection{General optomechanics}
%\begin{itemize}
%\setlength\itemsep{0.1em}
%\item One \textbf{XT66DP-1000} optical rail and ten \textbf{XT66C4} clamping platforms
%\item Lens holders: four \textbf{LMR2}, three \textbf{CP02}
%\item Two packs of post holders \textbf{PH50-P5}.
%\item Post packs (one of each): \textbf{TR30/P5}, \textbf{TR40/P5}, \textbf{TR50/P5}.
%\item Screen or filter holder: \textbf{DH1} (buy two if you skip the translatable slide holder)
%\item Translatable slide holder \textbf{XYFM1} (optional)
%\item Linear translation stage and cage rods for focusing: \textbf{SM1Z} and four \textbf{ER2} posts (optional)
%\item \textbf{RA180} and \textbf{RA90} post clamps.
%\end{itemize}
%
%
%\subsubsection{Optics}
%\begin{itemize}
%\setlength\itemsep{0.1em}
%\item LED Collector lens: 1'', $f=25 mm$ \textbf{LB1761} or $f=30~mm$ \textbf{LA1805}
%\item Field Lens/Condenser Lens: 2'', $f=60 mm$ \textbf{LA1401} (two)
%\item $f=100~mm$ \textbf{LA1050}, $f=300 mm$ \textbf{LA1256}, $f=200 mm$ \textbf{LA1979}
%\item $f=300~mm$ achromatic doublet (\textbf{AC508-300-A}). Optional.
%\item $f=-50 mm$ \textbf{LC1715}
%\item Olympus 4X objective (\textbf{WD RMS4X}) and RMS to SM1 coupler (\textbf{SM1A3}).
%\end{itemize}
%
%\subsubsection{Misc}
%\begin{itemize}
%\setlength\itemsep{0.1em}
%\item Two post mounted iris diaphragms (\textbf{ID25}). 
%\item Epoxy-Encased LEDs, 525 nm, 7 mW (\textbf{LED528EHP}).
%You will need to chop off the rounded end of the LED, so it doesn't act like a lens.  
%A razor blade and very fine polishing paper work for this. Alternatively, find bright non-domed (flat) LEDs.
%\item LED mount \textbf{S1LEDM}
%\item A 1'' SM1 tube (\textbf{SM1L10}) to make it easier to place the 25 mm lens near the LED.
%\item A power source for the LED. e.g. a 5V source and $200\Omega$ resistor. 
%\item Coverslips and a marker pen.
%\item Slides to image. Golgi stained brain slices work well. 
%Failing that, you could by a `toy' slide kit like that supplied by Celestron. 
%Electron microscopy grids of known pitch might be useful but aren't explicitly used in these exercises.
%\item A tape measure and ruler. 
%\end{itemize}
%
%\subsubsection{Tools}
%\begin{itemize}
%\setlength\itemsep{0.1em}
%\item Lens wrenches for SM1 and SM2: \textbf{SPW606} and \textbf{SPW604}
%\item Hex keys such as \textbf{CCHK} or \textbf{TC2} (\textbf{TC3} for metric). 
%\item \textbf{LMR2AP} alignment target
%\item Four \textbf{R2} post collars
%\end{itemize}
%
%\subsubsection{In addition}
%In addition to the above, you will need the following items that can be shared across multiple set ups.
%\begin{itemize}
%\setlength\itemsep{0.1em}
%\item Cap screw kit \textbf{HW-KIT2}
%\item A laser pointer that fits into an RA180 post clamp that attaches to one end of a post.
%\item A bright light source, such as a halogen lamp and light-guide. 
%\end{itemize}
%
%\section{Equipment usage notes}
%The black post holders mount to the rail carriages via a short cap screw as shown in Fig.~\ref{fig:post}. 
%The carriages can then be attached to the rail, on which they can slide up and down. 
%Use the lens tool to secure lenses in lens holders. 
%`SM2' refers to two inch optics and associated accessories. 
%`SM1' refers to one inch optics and associated accessories. 
%
%\begin{figure}[h]
%\center
%\includegraphics[width=1.3in]{post_mounted.eps}
%\includegraphics[width=1.5in]{post_mounted_underside.eps}
%\includegraphics[width=2in]{lens_tool.eps}
%\caption{A post holder bolted to a rail carriage (left \& middle).
%The lens tool secures lenses in the holders using the rings (right).}
%\label{fig:post}
%\end{figure}
%
%
%\clearpage

\section{Image Formation}
An image is formed when all light rays leaving one point (or region) of the object arrive at some other defined point (or region) \textit{regardless of the angle of the ray}. 
This is illustrated in Fig.~\ref{fig:imageforming}, where the grey mouse on the left is imaged through a lens to form an inverted image: the green mouse on the right. 
All light rays leaving the grey mouse's left ear converge onto the inverted image's left ear. 
When we put a sheet of paper in the image plane, we see points on the paper illuminated by rays coming from corresponding points in the object. 



\begin{figure}[h]
\center
\includegraphics{image_forming_basics.eps}
\caption{Simple image formation using one lens. 
The focal length of the thin lens is $f$, object location is $s_o$ and image location is $s_i$. 
The upper ray (parallel to the optical axis on the left) passes through the focal point (denoted as a dot) on the right side of the lens, the middle ray (passing through the centre of the lens) is unrefracted, and the lower ray passes through the focal point on the left side of the lens, and comes out parallel to the optical axis on the other side. 
}
\label{fig:imageforming}
\end{figure}

You will need to get comfortable with drawing ray diagrams, since you will do this over and over again in the school.
Redraw the diagram in Fig.~\ref{fig:imageforming}.
Draw the three cardinal rays:
\begin{itemize}
\item The ray parallel to the optical axis goes through $f$ on the image side.
\item The ray that passes through the centre of the lens travels undeflected.
\item Finally, the ray that goes through $f$ on the object side leaves the lens parallel to the optical axis on the image side. 
\end{itemize}

As the location of the object with respect to the lens ($s_o$) is varied, the location of the image ($s_i$) also changes. This relationship is determined by the lens focal length ($f$) using the thin lens equation.

\begin{equation}
\frac{1}{f} = \frac{1}{s_i} - \frac{1}{s_o}
\label{eq:thinlens}
\end{equation}

The following conventions are used: locations to the left of the lens are negative and those to the right of the lens are positive, $f>0$ for convex lenses, $f<0$ for concave lenses.
Transverse locations above the optical axis are $>0$. Locations below are negative. 

\vspace{2.5em}



\clearpage




\subsubsection{1. Determine the focal length of a convex lens }
In the next few exercises, use the relationship described in \ref{eq:thinlens} to measure the focal length of a convex lens. You will use an LED as both a light source and an object to image. 
\vspace{1em}

\textbf{Do not exceed 0.5 A when powering the LED}.
\vspace{1em}

%You will form various images of the LED emitter. 
%Your first task is to set up the LED on the optical rail:
%
%\begin{itemize}
%\item Attach a post holder to a rail carriage (Fig.~\ref{fig:post}) and place this on the rail. 
%\item Place the LED in the SM1 holder and attach this to a cage plate (CP02) and mount this on a 50~mm post. 
%\item Mount the post to your post-holder and place the LED at one end of the rail.
%\item Power the LED (remembering to use the current-limiting resistor). 
%\end{itemize}


%Now choose a lens with a focal length of about $f=100~mm$ and attach the lens to a lens holder using the lens wrench and a retaining ring (this should already be in the lens holder).
%Mount the lens on another carriage and attach to the rail next to the LED. 
%Get the LED aligned with the middle of the lens (precise alignment isn't important). 

%You will use this lens to form an image of the LED.

You will be able to form an image on a piece of card or paper if the distance between the lens and LED is $>1f$. This is called \textbf{finite conjugate} imaging because the image is paired (i.e. conjugated) with an object at a finite distance from it.
The image and sample planes are said to be \textbf{conjugate planes}.
The image will be of the LED emitter, which usually looks like a small bright square, often with fine lines going across it. 
If all you see is a blur, you have not formed an image.

\begin{itemize}
\item Select a relatively long ($>100~mm$) focal length lens.
\item Place the lens $<1f$ from the LED and use a piece of card on the other side of the lens to verify that no image is formed at any distance from the lens.
\item Why is no image formed? 
Hint: draw a diagram with the object at $<1f$. Plot just two rays: the undeflected ray and the ray coming out parallel with the optical axis and then going through $1f$ on the image side. 
How do the rays behave after they have passed through the lens?
\item Verify that an image is formed when the lens is $>1f$ from the LED: place a card very far from the LED (e.g. at $20f$) and slowly move the lens away from the LED. 
What is the distance between the LED and the lens at which you see an image? 
It should be just over $1f$.
\item Measure $s_i$ for three values of $s_o$ and fill in the table below. 
Using equation~\ref{eq:thinlens}, calculate $f$ for each value of $s_o$.
\item What is the value of $s_i$ when $s_o=-2f$?
\end{itemize}

\vspace{2em}
\begin{tabular}{| p{1cm} | p{1cm} | p{1cm} |}
\hline
 $s_o$  &  $s_i$  &  $f$  \\
\hline
\hline
 & & \\ \hline
 & & \\ \hline
 & & \\ \hline
\end{tabular}

\vspace{2em}
Can you think of a quick method to estimate the focal length of a lens without doing any math? 

\subsubsection{2. Magnifying the image}
As you probably noticed, the size of the image of the LED emitter varied with $s_o$.
A lens produces images of different magnifications depending on $s_o$; an image can always be formed when $s_o>f$. 
The magnification of a lens is calculated as follows:

\begin{equation}
M = \frac{h_i}{h_o} = \frac{s_i}{s_o}
\label{eq:mag}
\end{equation}

A value of $M=1$ would mean unitary magnification (the image is the same size as the object). 
Negative numbers indicate an inverted image.

To begin thinking about magnification, hand-draw the ray diagram (as in Fig.~\ref{fig:imageforming}) for the smallest and largest values of $s_o$ that you used above. 
Make sure to give yourself plenty of room. 
Hint: You will run out of paper if you use values too close $1f$.
If you're stuck, choose $1.5f$ and $4f$.
Note the changing angles of the rays and how they lead to different image sizes as $s_o$ changes. 
%We will now see what happens when the rays come from infinity:
%
%\begin{itemize}
%\item Mount a lens of about $f=60~mm$ in a holder and attach a post to it. 
%Go over to the window and form an image of the outside world on a piece of paper. 
%The image is formed at $1f$.
%\item Repeat with a lens of about $f=100~mm$ lens. 
%What two things do you notice about the image? 
%\item Fig.~\ref{fig:outside} models the situation you saw. Satisfy yourself that this provides a reasonable explanation. 
%\item Go back to the lens and LED.
%Calculate the values of $s_o$ and $s_i$ which produce $M=-4$ (remember, negative just means inverted) for a $f=100~mm$ lens. 
%Measure the size of the LED image and so calculate the size of the LED emitter. 
%\item Remember the value of $s_i$ when $s_o=2f$? Therefore what is $M$ when $s_o=2f$?
%\end{itemize}
%
%
%\begin{figure}[h]
%\center
%\includegraphics[width=2.5in]{image_forming_outside.eps}
%\caption{Images formed at $1f$ from light originating at infinity. 
%The top is a $f=50~mm$ lens and the bottom a $f=200~mm$ lens.
%Light from different regions of the object arrive in parallel rays that come in at different angles. 
%Each set of parallel rays come into focus at a point, satisfying the image-forming condition. }
%\label{fig:outside}
%\end{figure}


%What you have covered above demonstrates an important property of lenses: that they work the same both forwards and backwards.
%Recall the definition of an image-forming condition: \textbf{light rays leaving one point (or region) of the object arrive at some other defined point (or region)}.
%This is why no image was formed when the LED was at $s_o=f$, since rays leaving the lens are parallel and do not converge on the other side. 
%However, you \textit{were} able to form an image at $s_i=f$ when the object was very far away. 
%The ray diagrams of these two conditions are \textit{identical}--the only difference is that the object is at $1f$ in the first case and at infinity in the second case.

\clearpage


\subsubsection{3. Virtual images}
At distances $s_o<f$, you were unable to form an image of the sample on the card because the rays emerging from the other side \textit{diverged}.
\begin{itemize}
\item Place the LED at $0.5f$ from a lens of your choice, observe the light diverging as it exits the lens.
\item Draw the ray diagram for this $0.5f$ scenario:
Draw the ray that leaves the tip of the object and continues undeflected through the middle of the lens. 
Draw the ray that travels parallel to the optical axis and goes through $1f$ on the other side.
See how these rays diverge. 
\end{itemize}

It would seem that no image is formed yet, surprisingly, a \textit{virtual image} is formed on the same side of the lens as the object. 
This may sound like a peculiar concept, but you are already familiar with virtual images as this is how you see images in flat mirrors (Fig.~\ref{fig:mirror}). 
The extended rays in the case of the mirror can be created from the \textit{diverging} rays from the object. 
In the case of the the lens at $s_o<f$ we also have diverging rays and so can also draw extended rays as we did for the mirror. 
Add the extended rays to your ray diagram by continuing the rays on the object side of the lens until they converge. 
A virtual image is formed where they converge.
What does your diagram tell you about the magnification of the virtual image?
Verify this: pick up the lens, place it $<1f$ from an object and look through it.
\begin{figure}[h]
\center
\includegraphics[width=4in]{virtual_image_mirr.eps}
\caption{A virtual image of the bottle is created on the far side of the mirror. 
This is described by the dotted lines which are `extended rays' that are used to form the virtual image. }
\label{fig:mirror}
\end{figure}

A virtual image is also formed by a negative (concave) lens, as shown in Fig.~\ref{fig:neglens}. 
Examine the ray diagram of the negative lens. 
What do you predict you will see if you look through a negative lens?
Verify this with the $f=-50~mm$ lens.
\begin{figure}[h]
\center
\includegraphics[width=3.5in]{negative_lens.eps}
\caption{Ray diagram of a negative lens. Note that the image is not a reflection, but a virtual image. }
\label{fig:neglens}
\end{figure}

%In summary, a \textbf{real image} is formed on the opposite side of the lens to the sample. 
%This allows the detector to be physically separated from the sample, which is rather useful.
%A \textbf{virtual image} is formed on the \textit{same} side of the lens as the object. 

\clearpage

\subsection{Infinite Conjugate}
The \textbf{infinite conjugate} refers to the situation where the object is located at a distance of $1f$ from a lens (Fig.~\ref{infiniteConjugate}). 
In this scenario the rays leaving the lens are parallel and do not converge, so this is not an image forming condition. 
Instead, the image can be considered to be infinitely far away (hence the name). 
If a second lens is added, then an image can be formed. 
The magnification of the image is defined as Eq.~\ref{eq:magIC}. 
This arrangement is useful because the image is always formed at $1f$ from the second lens irrespective of the distance between the lenses.
Most microscope objectives are designed to work in this configuration since it allows filters to be added into the infinite space without altering the location of the image plane. 
Such objectives are known as `infinity corrected`, since they are designed to produce their best images with the sample at $1f$.

\begin{equation}
M=-\frac{f_2}{f_1}
\label{eq:magIC}
\end{equation}

\begin{figure}[h]
\center
\includegraphics[width=4in]{infiniteConjugate.eps}
\caption{The infinite conjugate.}
\label{infiniteConjugate}
\end{figure}

\begin{itemize}
\item Set up two lenses with different focal lengths on the rail to build the infinite conjugate. 
Ensure that the first lens is at $1f$ from the LED before placing the second lens. 
\item Place a screen at the $f$ of the second and hold it in place with the post-mounted clip.
\item Verify the consequences of the infinite space by moving the second lens and maintaining the screen at $f$ of the second lens. 
What happens to the image size?
\item Swap the first lens with one of a different focal length. 
Verify that the image size changes in the manner in which you expect. 
\end{itemize}



\clearpage


\subsection{Beam expanders}
Lenses can be used to expand the diameter of a light beam, such as a laser.
Expanders can be built using either two convex lenses (Fig.~\ref{beamExpander1}) or a convex and concave lens (Fig.~\ref{beamExpander2}). 
In both cases the lenses are set up such that their focal points coincide (i.e. they are separated by the sum of their focal lengths). 
The image formed by the first lens is imaged at infinity by the second lens.
The degree to which a beam is expanded is given by:
\begin{equation}
\frac{\phi_2}{\phi_1}=\frac{f_2}{f_1}
\label{eq:beamExp}
\end{equation}
where $\phi$ is the beam diameter.

\begin{figure}[h]
\center
\includegraphics[width=4.5in]{beamExpander1.eps}
\caption{Beam expander with two convex lenses.}
\label{beamExpander1}
\end{figure}

\begin{figure}[h]
\center
\includegraphics[width=4.5in]{beamExpander2.eps}
\caption{Beam expander with one convex and one concave lens.}
\label{beamExpander2}
\end{figure}

We need collimated light (parallel rays) entering the system, so swap the LED for a laser pointer. Uses irises to make sure that the laser beam is parallel to the optical rail before you start mounting the lenses.

Build a 2x beam expander with two convex lenses. Place them at the sum of their focal lengths and use the laser pointer and irises as a guide for placing the lenses. 

\begin{itemize}
%\item Place two more rail carriages onto the rail, between the iris and the laser pointer. 
%\item Place the $f=100~mm$ lens onto the post carriage nearest the laser and position the iris roughly $1f$ from the lens.
\item You will know the lens is at the correct height if the beam hits the middle of the iris you previously aligned with the pointer. 
You can also use the back-reflection of the beam from the lens to check whether the lens is aligned. 
%\item Slide the iris down the rail and complete the expander with the $f=300~mm$ lens. Again, use the iris as a target to judge lens height.
\item Make sure that the two focal points coincide by checking that the exit beam remains collimated (does not diverge or converge) at all distances from the second lens. 
%\item Measure the size of the expanded beam and so calculate size of the unexpanded beam.
\item Build a beam expander with the same magnification with a concave and a convex lens. What is the advantage of this configuration?
%What advantages does the negative lens add?
%\item Build a beam expander to yield the maximum magnification your optics kit allows (e.g. $f=300~mm$ and $f=30~mm$ to yield 10x). 
%The beam expander you have built also goes by another common name.  What is it? 
%Hint: think about what happens to beams that do not travel on-axis (Fig.~\ref{fig:telescope}).
%Once you've figured it out, remove the laser pointer and use your device. 
\end{itemize}

%
%\begin{figure}[h]
%\center
%\includegraphics[width=4.5in]{telescope_ray_diag.eps}
%\caption{A beam expander composed of an $f=400~mm$ lens and an $f=40~mm$ lens. 
%In addition to the collimated beam arriving on-axis, two off-axis beams are shown.}
%\label{fig:telescope}
%\end{figure}
%
%\clearpage
%
%
%
%\subsection{Composite optical elements}
%As you will have noticed above, image quality tends to be best in the centre of the field of view and decreases towards the edges. 
%A variety of aberrations (Fig.~\ref{fig:aberrations}) are at play and these tend to become progressively worse for light coming in at steeper angles to the optical axis.
%
%\begin{figure}[h]
%\center
%\includegraphics[width=6in]{aberrations.eps}
%\caption{The three most common optical aberrations. 
%Chromatic aberration sees light of different wavelengths coming into focus at different distances from the lens.
%Spherical aberration sees on-axis light coming into focus at different distances from the lens. 
%Coma is the situation where off-axis light does not come into focus in the same spot. }
%\label{fig:aberrations}
%\end{figure}
%
%Aberrations are corrected in objectives and eyepieces (Fig.~\ref{fig:composite}) by combining optical elements with complementary aberrations. 
%Note that in objectives and eyepieces the light doesn't pass through a focal point within the device and so they can be considered to be a single complex lens.
%
%\begin{figure}[h]
%\center
%\includegraphics[width=2.8in]{objectivesfigure1.eps}
%\includegraphics[width=2.2in]{eyepieces5.eps}
%\includegraphics[width=2.2in]{Plossl.eps}
%\caption{The composite lens arrangements found in microscope
%  objectives (left) and eyepieces (middle).
%  The Pl\"{o}ssl eyepiece is composed of two sets of achromatic doublet lenses (bottom).}
%\label{fig:composite}
%\end{figure}
%
%
%Here we will experiment with a simple composite optical arrangement: the Pl\"{o}ssl eyepiece (Fig.~\ref{fig:composite}). 
%Pl\"{o}ssls are commonly used in amateur astronomy because they are not expensive and provide a relatively wide (50 degree) field of view that is of good 
%quality if the eyepiece is well made. 
%Pl\"{o}ssls also make suitable scan lenses for two photon microscopy\footnote{Negrean \& Mansvelder, 2014, PMID: 24877017}.
%You will now construct a Pl\"{o}ssl using two $f=60~mm$ singlet lenses.
%
%
%\begin{itemize}
%\item First construct a beam expander ($f_1$=30~mm lens and a $f_2$=300~mm). 
%The expanded beam will be directed into your eyepiece, which will make it easier to measure its focal length.
%\item Build the Pl\"ossl by placing the two $f=60~mm$ lenses as close as possible with their flat sides facing outwards. 
%You might need to move the post holders on the rail carriages to allow the lenses to meet.
%Since collimated light leaves your beam expander, you can place the Pl\"ossl at any distance from the $f=300~mm$ lens. 
%\item Verify that your Pl\"{o}ssl behaves more or less as expected: 
%Eq.~\ref{eq:compoundLensF} gives the distance between the middle of the compound lens and the focal point. 
%\item Separate your lens elements by about 15~mm and measure and measure the change in focal length.
%\end{itemize}
%
%The effective focal length of two thin lenses separated in air by some distance $d$ is given by
%\begin{equation}
%\frac{1}{f} = \frac{1}{f_1} + \frac{1}{f_2} - \frac{d}{f_1f_2}
%\label{eq:compoundLensF}
%\end{equation}

%A telescope built from two positive lenses is known Keplerian telescope. 
%A Galilean telescope uses a negative lens as the eyepiece and forms a non-inverted image.
%Build a Keplerian telescope using $f=300~mm$ singlet lens for the objective and a $30~mm$ lens for the eyepiece. 
%Look through it familiarise yourself with what the image looks like then replace the eyepiece with your Pl\"{o}ssl. 
%The magnification will be similar, but the image quality should be better: you'll get a wider apparent field of view and substantially less pincushion distortion. 
%You will still see chromatic aberration. 
%If you have an achromatic $f=300~mm$ doublet, then you can use this as the objective to get a reduction in chromatic aberration and a sharper image. 
%Your telescope is powerful enough to see the moons of Jupiter. 

\clearpage
\subsection{Optics Challenge}
Arrange 4 lenses so as to form a de-magnified, real, upright
image. Hints: 
\begin{itemize}
\item Use one negative lens.
\item Space will be a problem: the lenses will take up the whole rail
  and so you will need to place the target and screen outside of the
  rail. 
\item Use the light-guide to illuminate your target to avoid it
  becoming too dim. 
\item There are multiple solutions to this problem.
\end{itemize}


\clearpage
\section{Illuminating the sample}
Different samples are best illuminated in different ways.
In transmission microscopy, the sample is illuminated from the opposite side to which it is imaged. 
In fluorescence microscopy, the sample is illuminated from the same side, with the objective also serving as the condenser. 
It is not efficient to illuminate the sample directly (with no intervening lenses) because light is emitted in all directions from the source but only a narrow range of ray angles reach the specimen. 
In the following two exercises you will build two different illumination systems: critical and K\"{o}hler. 
%You will be given a brighter light source (either a bright blue LED or a halogen light-guide).


\subsection{Critical illumination of a sample}
In critical illumination the light source is focused onto the specimen (Fig.~\ref{critIlum}).
A suitable sample is a coverslip with thin lines drawn on with a marker.
Set up critical illumination with a $f=25~mm$ lens. Use the 4x objective and a $f=300~mm$ (1") lens to form an image of the sample on a card.

\begin{figure}[h]
\center
\includegraphics[width=5in]{Critical_Illumination.eps}
\caption{A simple critical illumination set up for visual microscopy.}
\label{critIlum}
\end{figure}


%\begin{itemize}
%%\item Hint: to avoid running out of $50~mm$ posts in the exercise that follows, use $30~mm$ or $40~mm$ posts for the 2'' lenses, sample holder, and card holder.
%%\item Place the light source at one end of the rail and then put an $f=25~mm$ or $f=30~mm$ 1'' lens at $2f$ from it.
%\item Place your sample at $2f$ on the other side, where the image of the LED emitter or fiber bundle is formed.
%Note that you can place a glass slide in a DH1 card holder (the black clip) by removing a clip on one side and loosening the screws a bit.
%\item Move the slide back and forth until an image of the LED emitter appears on it. Use paper to help you see this if needed.
%\item Place the 4x objective after the slide. This objective has a working distance of about $20~mm$.
%\item At the other end of the rail, place a viewing card onto which you will form an image.
%\item Position an $f=200~mm$ lens $1f$ from the card. This is known as the tube lens and will image the sample onto the card
%\footnote{The $f=200~mm$ lens will yield 4x. You could use the $f=300~mm$ lens if you wish, to yield 6x.}.
%\item Move the objective back and forth a small distance to focus and obtain an image. 
%\end{itemize}

The above arrangement is a simple microscope. 
As you might expect, the image of the sample also contains an image of the light source. 
This is obviously not ideal.
Before we go on to address this problem, let's improve the illumination by capturing more of the light from the source. 
We can do this by placing the $f=25~mm$ lens closer to the light source and adding an $f=60~mm$ lens after the $f=25~mm$ lens, forming an infinite conjugate system. 
The lens nearer the source is known as the \textbf{collector lens} and will be at $1f$ from the source. The position of the focal plane can be difficult to estimate for short focal length lenses. Ask you favourite TA for pointers.
The lens nearer the sample is known as the \textbf{condenser lens} and is at $1f$ from the sample. 
This is still critical illumination, since the light source will be imaged onto the sample, but the collector lens will be closer to the source and so more light will be captured.

%\begin{itemize}
%\item Remove all the posts (not the rail carriages) from the rail.
%\item Add all 10 post and rail carriage assemblies to your rail. As you proceed, keep in mind that you will need 6 carriages between the objective and the light source.
%\item Replace the light source with the laser pointer on a $75~mm$ post. 
%\item Use an iris as before to align the beam. However, this time you will have to move it between post holders and maintain its height with an RA90 (90 degree clamp) 
%attached to the iris post (Fig.~\ref{fig:clampedIris}) since you won't be able to slide it up and down the rail.
%\item Replace the tube lens and position it at the same height as the beam. Use the iris to help you.
%\item Add objective upstream of the tube lens. Get the laser beam hitting the middle of the of front element. 
%\item Now place the iris in the specimen's post holder and locate the $f=60~mm$ condenser lens next to it, using the iris as a target. 
%\item Replace the iris with the sample
%\item Use the iris to place the $f=30~mm$ collection lens in front of the laser.
%\item Leave the iris where it is but open it. 
%\item \textbf{You should have two empty carriages between the iris and the condenser lens}
%\item Replace the laser pointer with your light source. 
%\item Move the lenses back and forth so as to obtain a focused image under critical illumination. 
%\end{itemize}


%\begin{figure}[h]
%\center
%\includegraphics[width=3in]{clamped_iris.eps}
%\caption{A clamp used to maintain the iris height so it can easily be moved between post holders}
%\label{fig:clampedIris}
%\end{figure}

With an infinite conjugate illumination system it becomes possible to regulate the NA of the illumination by varying the size of the iris in the infinite space. 
Try it!
By altering the NA of the illuminated light, the iris provides control over the resolution, contrast, and depth of field.
To see the effect: open the iris, defocus slightly by moving the objetive, now close the iris. The image will go back into focus. 
You could also draw a cross on a coverslip with a marker and lay over your sample, then open and close the iris.
%
%\begin{figure}[h]
%\center
%\includegraphics[width=2.8in]{critical_open_iris.eps}
%\includegraphics[width=2.8in]{critical_closed_iris.eps}
%\caption{Critical illumination with two lenses. 
%There is an iris after that collector lens that is open on the left image and closed on the right image. 
%Note how closing the iris restricts the range of angles reaching the sample.}
%\label{fig:critical_iris}
%\end{figure}



\clearpage

\subsection{K\"{o}hler Illumination}
K\"{o}hler illumination is an important and commonly used technique in light microscopy, as it provides even illumination of the sample whilst ensuring the light source (e.g. the bulb filament) is not visible. 
This is achieved by adding a third lens, the field lens, before the condenser so that the image of the light source is out of focus at the sample. 
Since the source is out of focus, the illumination is uniform. 
Instead, the image is formed at $1f$ from the field lens. 
The condenser is then located at $1f$ from this image (as in a beam expander) so collimated (i.e. out of focus) light reaches the sample. 

The key to understanding K\"{o}hler illumination is to consider the two sets of conjugate planes in the system (Fig.~\ref{fig:koehler}). 
The \textbf{sample} is, of course, conjugate with the image plane but it also conjugate with a point $1f$ from the collector lens ($f_{CL}$ in Fig.~\ref{fig:koehler}). 
The \textbf{light source} is conjugate with a location between the field lens and the condenser ($f_F$ and $f_{CO}$ in Fig.~\ref{fig:koehler}). 
The light source is also conjugate with the objective back aperture (not shown in Fig.~\ref{fig:koehler}). 

\begin{figure}[ht]
\center
\includegraphics[width=5in]{koehler.eps}
\caption{K\"{o}hler Illumination. 
The blue lines indicate the conjugate relationship between the sample plane and the location of the field diaphragm.
The yellow lines indicate the conjugate relationship between the light source plane and the location of the aperture diaphragm.
}
\label{fig:koehler}
\end{figure}

At the the two conjugate planes in the illumination path are two irises (diaphragms). 
The \textbf{field diaphragm} is at $1f$ before the field lens, in the point conjugate with the sample. 
The field lens and condenser lens form an infinite conjugate system that image the field diaphragm onto the sample. 
The field diaphragm appears in focus at the sample and is used to regulate the area of illumination. 
This helps to improve contrast by minimizing stray light. 
Remember, the The \textbf{f}ield planes are the image \textbf{f}orming planes. 

The \textbf{aperture diaphragm} is located in the plane conjugate with the light source. 
i.e. it is located where an image of the light source is formed. 
Opening and closing this diaphragm regulates which regions of the light source can contribute to the illumination. 
Closing the diaphragm blocks off-axis regions of the light source and stops them from generating the more oblique rays that come out of the condenser. 
Closing the aperture diaphragm reduces the NA of the illumination and increases depth of field. 

\clearpage

You will now convert the critical illumination to K\"{o}hler illumination simply by adding the $f=100~mm$ field lens and an extra iris.
The components will be arranged in the following order and are all at $1f$ from each other (Fig.~\ref{fig:koehler}):
\begin{enumerate}
\setlength\itemsep{0.1em}
\item Collector lens ($f=25~mm$)
\item Field diaphragm.
\item Field lens ($f=100~mm$)
\item Aperture diaphragm
\item Condenser lens ($f=60~mm$)
\item 4x objective
\item Tube lense ($f=~300mm$).
\end{enumerate}

%The completed arrangement of the components is shown in Fig.~\ref{fig:koehler_completed}. 
Build the K\"{o}hler setup and observe that the image of the emitter is gone. 
You will likely need to tweak the distances between the components to get the field diaphragm conjugate with the sample. 
Once this is done, regulate the size of the field diaphragm and observe the effect. 
Defocus slightly and close the aperture diaphragm. Observe the effect.
You have finished! But our princess is in another castle! See you tomorrow. 

%\begin{figure}[h]
%\center
%\includegraphics[width=5.5in]{illum_complete.eps}
%\caption{The completed K\"{o}hler illumination assembly on a rail.}
%\label{fig:koehler_completed}
%\end{figure}






\end{document}
