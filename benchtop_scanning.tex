%%%%%%%%%%%%%%%%%%%%%%%%%%%%%%%%%%%%%%%%%%%%%%%%%%%%%%%%%%%%%%%%%%%%%%%
%%%%  Load the document class and packages                         %%%%
%%%%%%%%%%%%%%%%%%%%%%%%%%%%%%%%%%%%%%%%%%%%%%%%%%%%%%%%%%%%%%%%%%%%%%%
\documentclass[a4paper]{report}
\usepackage{epsfig}            % to insert PostScript figures
\graphicspath{ 
  {figures/} 
}

%Change figure names
\renewcommand{\figurename}{Fig}

\usepackage[bf,footnotesize]{caption} % make captions small and label bold


\addtocounter{chapter}{1} %Because starting at zero is silly
\makeatletter
\renewcommand{\thesection}{\@arabic\c@section}
\renewcommand{\thefigure}{\@arabic\c@figure}
\makeatother

\usepackage[a4paper,margin=2.7cm,tmargin=2.5cm,bmargin=2.5cm]{geometry} 
\usepackage{textcomp}          % To make nice degree symbols and others\usepackage[bf,footnotesize]{caption} % make captions small and label bold
\usepackage{wrapfig}
%to produce the clickable references along the left in Acroread. This
%package must be included last. 
\usepackage[ps2pdf,bookmarks=TRUE]{hyperref} 



%%%%%%%%%%%%%%%%%%%%%%%%%%%%%%%%%%%%%%%%%%%%%%%%%%%%%%%%%%%%%%%%%%%%%%%
%%%%  Hypertext references for Acrobat                             %%%%
%%%%%%%%%%%%%%%%%%%%%%%%%%%%%%%%%%%%%%%%%%%%%%%%%%%%%%%%%%%%%%%%%%%%%%%
\hypersetup{
pdfauthor = {TENSS},
pdftitle = {Benchtop Scanning},
pdfkeywords = {optics, lenses, refraction, reflection, dispersion,
  telescope, microscope},
pdfcreator = {LaTeX with hyperref},
pdfproducer = {dvips + ps2pdf}
           }


\begin{document}




%set the number of sectioning levels 
\setcounter{secnumdepth}{2}

\begin{center}
\textbf{\Large{Benchtop Scanning}}
\end{center}

\section{Introduction}

\subsection{Scanning microscopy}
Widefield imaging uses lenses to form a real image of the sample which is magnified and imaged onto a CCD chip or viewed by eye. 
In this scenario the sample is conjugate with the image plane. 
In widefield microscopy the entire field of view is illuminated so a great deal of scattered light from outside of the focal plane ends up in the image plane. 
Recall the definition of an image-forming condition: \textbf{light rays leaving one point (or region) of the object arrive at some other defined point (or region)}.
Scattered photons are so defined because they arrive at the image plane at a location \textit{not conjugate} with where they originally came from.
Thus, scattered photons fail to satisfy the image-forming condition. 
Scattering is due to the inhomogeneous refractive index of the sample. 
Photons that are not scattered are known ballistic photons. 

As you saw previously, image quality can be improved by an well set up K\"{o}hler illumination system.
However, even K\"{o}hler illumination does not fix the fundamental problems with wide-field imaging and it can only be used with thin samples where transmitted light illumination is possible.

Fluorescence-based scanning microscopy greatly mitigates the problem of scattering. 
A laser beam is scanned across the sample and excites a fluorophore. 
Emitted fluorescence is collected via the objective and detected with a photomultiplier tube (PMT). 
No image is formed on the PMT (it's a `single pixel'), instead the image is constructed \textit{post-hoc} on a computer from the time-series PMT data.
Indeed, in many scanning microscopes the object and the PMT are not even at conjugate planes.
The reason scattered light is less problematic in scanning microscopy depends on the design of the microscope:
With confocal, the scattered light is rejected by the pinhole. 
With 2-photon, the region of excitation is highly restricted so the origin of all collected photons is known. 




\subsection{Building a transmission scanning microscope}
Today you will build a transmitted light scanning microscope using a set of scanners, three lenses, two mirrors, and a laser pointer.
For detection you will use a photodiode located after the sample. 
This setup of course provides none of the advantages of a scanning microscope, since it's not fluorescence based and there is no pinhole. 
However, the arrangement of the excitation path optics will be identical to that found in the 2-photon microscope you will go on to build. 

\section{Instructions}

\subsection{Set up the scanners and align the beam}
Set up the rail and scanners as follows:
\begin{itemize}
\setlength\itemsep{0.15em}
\item Bolt the optical rail to the breadboard using a clamp on each end. Do not over-tighten. 
\item Place two irises on the ends of the rail and set them to the same height. 
Locating them on fully lowered $75~mm$ posts is suitable. 
\item Place the the scanners at one end of the rail and power them (ask for assistance).
The scanners need to be powered for the mirrors to be square with respect to each other. 
\item Loosely bolt the scanners into place. 
\item You may end up needing to the move the rail or the scanners during the alignment procedure.
\end{itemize}

\vspace{1.5em}

You're now ready to position and align the laser. 
You will use two mirrors to align the beam going into the scanners.
Your goal is to have the beam hit the middle of the scan mirrors then go straight down the rail, parallel with the table. 
This will be easiest if you position one mirror, angled at 45 degrees, close to the scanners.
The next mirror should be some distance from the first one and square with respect to it. 
The laser pointer is placed at 45 degrees with respect to this second mirror. 
Do the initial coarse alignment by moving and rotating the mirrors and the laser pointer. 
Then bolt down the components and use the fine-adjustment screws on the mirror mounts. 
Use the irises to ensure the beam runs straight down the rail. 
You will likely find that the scanners aren't aligned with the irises and that either they or the rail need to be moved. 
Expect to do the alignment procedure several times. 
It's OK for our purposes if the alignment isn't \textit{perfect}. 
Getting it to within a couple of beam diameters should be adequate. 


\subsection{Add the optics}
You will use the 4X objective to image the scan pattern onto the sample using the objective's full NA. 
Two things need to be achieved in order to make this possible. 
\begin{enumerate}
\setlength\itemsep{0.1em}
\item The scan mirrors will need to be in a conjugate plane to the back aperture of the objective. 
\item The beam will need to be expanded to fill the back aperture of the objective. 
\end{enumerate}

Both of these things can be satisfied by building a beam expander between the objective and the scanners. 
Place the objective mid-way down the rail and align it with the beam. Use an iris to guide you.
Choose suitable lenses for your beam expander and locate them in the correct places 
(think about what distance each element should be with respect to the others). 
You can move the irises to help you position the lenses. 
The lens nearer the scanners is called the \textbf{scan lens}.
The lens nearer the objective is called the \textbf{tube lens}.

\subsection{Verifying the scan pattern}
\begin{itemize}
\item Connect the scan control cables to the analog outputs of your acquisition card. 
The $x$ mirror is responsible for the `fast' axis and goes to AO0. The $y$ mirror is responsible for the slow axis and goes to to AO1. 
Ask if you are unsure how to do this. 
\item Start NI MAX, go to the test panels for your DAQ device and select analog output. 
\item Play a $10~Hz$ sine wave of amplitude $3~V$ through each channel in turn. 
\item Use a card or piece of paper to observe the beam motion through your optical system. 
What do you see at $1f$ from the scan lens?
What do you see at the working distance of the objective?
What do you see on the back of the objective?
What about in front of the objective?
Satisfy yourself that all those things make sense.
\end{itemize}


\subsection{Obtaining images}
\begin{itemize}
\item Place a slide containing an EM grid at the working distance of the objective. 
\item Place a photodiode close to the back of the slide and hook it up to AI0 on your DAQ. 
\item You may get a better image with a collection lens in front of the detector, but this isn't critical\footnote{Or even K\"{o}hler, for that matter.}.
\item If you have been introduced to the scan software, start the `minimal' version to begin scanning and acquisition.
The brave can write their own scan software from scratch. 
You can do it with MATLAB's Data Acquisition Toolbox in  about 150 lines of code. 
\item You will likely need to focus carefully to get an image: the depth of field is quite narrow. 
\item Your image will look distorted. 
Where does the distortion come from?
\item Modify the acquisition software to either (\textbf{a}) get rid of the artifact or (\textbf{b}) implement faster bidirectional scanning and correct the artifacts you see. You choose. 
\item Modify the software to save the image stream to disk. Hint: \textbf{help imwrite}
\item Acquire images of the grid at three different scan amplitudes and calculate the number of microns per pixel. 
\item Image a biological specimen, save the image, add a scale bar.
\end{itemize}


\end{document}
